% XeLaTeX can use any Mac OS X font. See the setromanfont command below.
% Input to XeLaTeX is full Unicode, so Unicode characters can be typed directly into the source.

% The next lines tell TeXShop to typeset with xelatex, and to open and save the source with Unicode encoding.

%!TEX TS-program = xelatexmk
%!TEX encoding = UTF-8 Unicode

\documentclass[authoryearcitations]{UoYCSproject}
\author{Andrew Durant}
\title{Self-Healing in Wireless Sensor Networks}
%\date{}
\supervisor{James Harbin}
\CSESE
\wordcount{0}

\includes{}

\excludes{}

\abstract{Abstract}

\dedication{}

\acknowledgements{}

% More definitions & declarations in project-report.ldf

\begin{document}
\maketitle
\listoffigures
\listoftables
\renewcommand*{\lstlistlistingname}{List of Listings}
\lstlistoflistings

\chapter{Introduction}
\label{cha:Introduction}

\section{Project Aims}

\section{Report Structure}

\section{Statement of Ethics}

\chapter{Literature Review}
\label{cha:LitReview}

% Introduction

In order to consider self-healing in real-world applications of wireless sensor networks an understanding of the possible applications is required. A survey of approaches to self-healing is then presented along with metrics that can be used to evaluate and compare them. This is followed with a review of evaluation techniques and real-world factors for wireless sensor networks.

\section{WSN applications}

The possible applications for wireless sensor networks are diverse and far reaching. The small self-powered devices that can gather data from a wide area, to detect events and communicate information back to a base station either processed or for processing lend themselves to a large number of applications. Each of these applications utilise the WSN in different ways and thereby bring new challenges to the developer of the systems. Due to this the design of the network is highly dependent on the application domain. Just some of the fields WSNs have been used in are environmental monitoring, warfare, agriculture, surveillance, medical care, education and micro-surgery.

Habitat monitoring using WSNs has been taking place on an island of ducks where ordinary monitoring techniques cause disturbance to the animals with problems such as increasing mortality and eventual abandonment of the habitat. The use of WSNs that can be deployed prior to seasonal habitation enables constant monitoring over the time without the need to disturb the inhabitants. \citet{Mainwaring2002} detail the requirements of the network and how it has been developed. Longevity of the devices, remote management, and inconspicuous operation are required to ensure the WSN is reliable and the data is able to be collected. The locations used for WSNs are often extreme, particularly for habitat monitoring. Similar requirements to the above are generated by \citet{Biagioni2002} for monitoring endangered plants. Most importantly however for collecting scientific data is ensuring the system is reliable and the user can have confidence in the data.

A very different approach was taken by \citet{Juang2002} in developing ZebraNet. Rather than statically positioned nodes, the devices are attached to the herd of zebra being monitored with collars. The nature of this system is entirely mobile, without even a fixed base station to report to. Because of this the nodes pass data between them when within range forming a distributed collection and storage network. The mobile base station can then be deployed occasionally to collect the dataset only needing to meet a few devices to obtain all of the data gathered by the network. This network is inherently ad-hoc where nodes may or may not come into contact with one another frequently, and there is no fixed or permanent base station to report to. Again longevity is an important requirement as deploying or re-deploying the sensors requires capture of the animals, so the systems must run for at least a year with no intervention.

Environmental monitoring on volcanoes has been investigated by \citet{Werner-Allen2006} where the use of WSNs enables data collection over a wider area for a longer amount of time compared to using traditional equipment. WSN nodes can be deployed easily and left unattended to monitor the volcano area over a period of several weeks or months. Whereas traditional equipment is bulky and heavy, and therefore difficult to deploy limiting the area of research. The WSN transmitting the collected data back to the base station also removes the need to frequently return to the sensor devices to retrieve data. This is a clear example of using the advantages of wireless sensor networks. However a number of potential issues are raised. The nodes are deployed at the limits of their range to the nearest neighbour, this allows as large an area as possible to be covered, however if a single node fails all other nodes that were dependent upon it will be unreachable. In the scale of this network, where the maximum hop length from the base station is 6 this may seem insignificant, however that is 37.5\% of the network nodes.


\section{Approaches to self-healing in WSNs}



\subsection{Centralised}

The main way centralised approaches are used in WSNs is in the configuration of the network structure to minimise energy use and ensure good coverage of the area of interest \citep{Wang2003,Ding2005,Wang2005,Derr2013}. Centralised evaluation of the network and area of interest prior to deployment ensures the connectivity and coverage requirements of the application are met.
\begin{quote}
Coverage requires that every location in the sensing field is monitored by at least one sensor. Connectivity requires that the network is not partitioned in terms of nodes' communication capability. Note that coverage is affected by sensors' sensitivity, while connectivity is influenced by sensors' communication ranges.
\citep{Wang2005}
\end{quote}
These vary by application for instance, detection and localisation of events require multiple nodes covering smaller areas (a dense coverage) whereas other applications may only need one node within its transmission range (a sparse coverage). By pre-planning to match the deployment to the application needs the network will be much more efficient in power use and accurate in data collection.

Networks with a high degree of connectivity are more resilient to node failures. \Citet{Wang2003} investigate the relationship between coverage and connectivity to produce an integrated solution that is able to maintain both requirements whilst reducing energy consumption. The degree of connectivity is described as $K$-connected, where any $K-1$ nodes may fail without loosing network connectivity. So the higher the connectivity ($>K$) the more nodes are able to fail without disrupting the rest of the functioning network.

Due to the differences in applications a number of application specific methods have been proposed in addition to those previously mentioned \citep{Meguerdichian2001,Meguerdichian2001a,Meguerdichian2003}. All of these provide for different application requirements in different ways. However a single parametrised method that can be applied to many differing applications has been developed by \citet*{Derr2013}. This allows the specification of the number of nodes and the distance between the nodes which defines the desired coverage and connectivity respectively. The algorithm can then take a mesh of the area and simplify it to provide the desired parameters.

Most centralised approaches use a priori information to specify exact deployment in the area of interest. However \citet*{Qu2012} have developed a post deployment particle swarm optimization (PSO) algorithm that is computed centrally to provide optimal coverage and reduce energy consumption. A centralised algorithm is able to understand the entire network, available devices, and area, and therefore produce the optimal distribution of nodes. The nodes are on a mobile platform that is able to move within the area. Reduction of energy consumption is done by dynamically reducing the range of sensing and communication on each device to the minimum needed once the locations have been determined. \citeauthor*{Qu2012} do not address node failures, as they assume all node are working throughout the time of use.

\subsection{Distributed}

In \citet{Abbasi2007}'s seminal paper DARA (a Distributed Actor Recovery Algorithm) in introduced. Developing a localised scheme to restore a network of mobile nodes when it has been partitioned by a node failure. The distributed nature of the algorithm avoids involving every single node, only requiring the local nodes to respond. The algorithm selects a node to replace a failed node and coordinates the local nodes to relocate to rebuild the network. As nodes relocate they may cause other partitioning, but this is dealt with in the same way forming a chain reaction until the entire network is reconnected. The self-healing process requires no supervision or centralised control.

There have been many algorithms developed for distributed control and self-healing for WSNs as a large amount to academic research has gone into systems such as environmental monitoring. A basis for many of these are DARA \citep{Abbasi2007} and PADRA \citep{Akkaya2008} which both aim to restore connectivity to a local area after a node has failed, whilst minimising the total distance travelled, and without external supervision or involvement.

RIM \citep{Younis2010} improves upon DARA and PADRA which each need 2-hop knowledge of the network to only requiring 1-hop knowledge on each node. This significantly reduces the network overhead for maintaining the required knowledge of the network topology, however the simplicity means that the distance travelled by the nodes is greater in larger networks for RIM than for DARA or PADRA. Calculating and transmitting a lot of detailed information is often considered too much overhead for low-power, low-complexity systems, particularly if the network can change often or easily. Depending on the application for the WSN the communication overhead to maintain the topology data could be justified over the generally more efficient algorithm.

Another flaw in RIM is that it assumes and requires only one node failure at a time. Whilst this may be the case, failures in WSNs are most commonly battery depletion, which is likely to occur at similar times across the network, or random failure due to environmental conditions, which could happen at any time to one or multiple nodes.

SFRA \citep{Alfadhly2012} is designed specifically to deal with multiple simultaneous failures to combat the multiple failure problem in RIM. Network trees are built from the root node, with local cluster-head nodes, this introduces a fair amount of network overhead compared with RIM. The number of updates needed to send is reduced by waiting for all child node messages before propagating back up the tree.

Distributed approaches become much more efficient as the network scales because they are only concerned with the local nodes that are directly reachable. This allows networks built upon distributed algorithms to become much larger, and cover much wider areas as the overhead increase from additional nodes is minimal compared with the exponential increase in complexity from the centralised approaches.

A number of biologically-inspired approaches exist utilising animal and insect coordination and inspired methods of load balancing. \citet{Caliskanelli2014} suggests the application of bee pheromone based load balancing to node distribution. The main thesis applies this algorithm to node redundancy enabling and disabling nodes within a dense area to preserve their life when they are unneeded. By deciding upon a cluster head for an area all other nodes move to a dormant state but coverage is maintained. If an active cluster head fails then a new cluster head will be assigned. Each cluster head emits a pheromone signal that prevents other nodes from becoming the head. To apply this to distribution, each node could emit a pheromone signal and move away from other signals within a maximum distance.

Another approach based upon the behaviour of ants in a colony has been introduced by \citet{Wang2014}. The system separates mobile nodes from the normal sensor nodes that are static. The network functions as normal on the static nodes. However if there is a need for relocation of a node, due to failure or any other cause, the mobile nodes can pick up the static nodes and travel with them to another location redeploying them there. This reduces the power and cost overheads of adding mobility to all of the nodes. The mobile nodes can carry a larger battery or travel to a charging station when not actively moving nodes.

A similar approach has been taken by \citet{Xu2015} for mobile nodes with high power capacity to travel between static sensor nodes, however the mobile nodes here are for recharging the static nodes. In most cases the reason for node failure is depletion of the battery, if this is the case, being able to recharge the sensor nodes in place increases the theoretical length of network use indefinitely.

\section{Metrics for evaluating self-healing WSNs}


\section{Review of evaluation techniques for WSNs}

When testing self-healing networks node failures must be induced, to do this a reliability function to control the failure rate of devices is needed. \citet{Zhou2004} investigate radio performance and integrity.

% move this para elsewhere
Most distribution / relocation algorithms require knowledge of the node location. Investigation into this is outside the scope of this paper, however there are many methods of obtaining location from GPS receivers to self-localisation algorithms that are well summarised by \citet*{Hu2004} and \citet*{Mao2007}.

% benefits and drawbacks of simulation / real-word factors affecting
% focus on this

\section{Conclusions}


\chapter{}

Wireless sensor networks are used in a wide range of applications, and in recent times this is only expanding. Due to their nature of being small, low-power devices, and the common network connectivity being multi-hop routing, network drop-outs and partitioning of devices is a common problem that has been tackled in a variety of ways. There are two main approaches to network and device management for combating and fixing these issues, the distributed approach and the centralised approach.

According to \citet{Tong2009} self-healing by means of mobile nodes still remains a greatly unstudied area.

The effectiveness of these two approaches is debated, however the application for a particular WSN determines the effectiveness of a particular algorithm or management paradigm. Certain use cases, for example in industrial equipment monitoring, power usage is less likely to be an important contributing factor but speed of detection and recovery might be more pressing. In an environmental monitoring situation network nodes are more likely to be physically difficult to get to once deployed, and longevity of battery power is highly important.

With these considerations, performance metrics for particular algorithms may not give a fair comparison, as algorithms optimised for low power consumption have a very different purpose to those optimised for rapid recovery. However the approaches and algorithms for communication in WSNs are still very different from traditional networking models because it is common for the network topology and availability to change often and quickly, the storage and network capacity is much lower, and wireless channels are prone to interference and drop-outs.


\bibliography{project-report}

%author = {\c{C}al\i\c{s}kanelli, \.{I}pek},

\end{document}
